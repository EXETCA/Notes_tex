\section{bash}
  \subsection{简介}
  \begin{itemize}
  	\item Unix shell is both ad command interpreter and a porgramming langguage.
  	\item 模式:交互模式与非交互模式。所谓非交互模式即shell从文件中读取命令并执行。
  	\item 内生的指令。
  		\begin{enumerate}
  			\item \textbf{cd} 
  			\item \textbf{break}
  			\item \textbf{continue}
  			\item \textbf{exec}
  		\end{enumerate}
  			.....
  	\item 语言基础:变量、控制流构建、引号、函数。
  	\item 特性:工作流控制、命令记录、别名。
  \end{itemize}
\subsection{语法环境}
	\begin{itemize}
		\item 注释符 \# 将剩下的一行注释
		\item Quoting 删除符号的特殊意义,有三种方法双引号、单引号、转义符
			\begin{enumerate}
				\item \textbackslash \quad 除了 \textbackslash newline 其他情况下满足上述
				\item '\quad 单引号 保持单引号内的(单)字符属性,即非特殊意义
				\item "\quad 双引号 保持(所有)字符属性,除了,\$ ' \textbackslash !这几个符号,当历史扩展使能时。 当Shell运行POSIX模式!,在并不能有特殊意义,即使开启历史扩展。\textbackslash 只有后面跟随 \$ ' " \textbackslash newline 时才具有特殊意义。在双引号中@ * 有特殊意义。
			\end{enumerate}
		\item  标准C的引用\\
		形式如同 \$'string',string 中有包含转义符的字符例如\textbackslash a 、\textbackslash b ....
		\item  本地特殊转换\\
		形式如同 \$"strings" ,根据本地环境不同将会对其作出不同处理。例如本地时C或POSIX环境,那么将会忽略\$符号并不进行转义。如果使用\textbf{shopt}的--noexpand\_translation 则转移字符双引号被单引号代替。通过bash --dump-po-strings scriptname > domain.pot 可以将显示脚本中转移后的文字效果。
		\item 注释\\
		除了\#是注释符外,在bash中可以通过textbf{shopt}的interactive\_comments 开启注释或关闭注释。
		\item 命令
			\begin{enumerate}
				\item 关键字\\
				\textbf{if} \textbf{then} \textbf{elif} \textbf{else} \textbf{fi} \textbf{time} \\
				\textbf{for} \textbf{in} \textbf{unitl} \textbf{while} \textbf{do} \textbf{done}\\
				\textbf{case} \textbf{esac} \textbf{coproc} \textbf{select} \textbf{function}\\
				\textbf {\{}  \qquad\textbf {\}} \qquad [[\quad]] \qquad !
				\item 管道\\
				| or |\& 符号 格式如下:
\begin{lstlisting}[numbers = none]
[time [-p]] [!] command1 [ | or |& command2 ] ...
\end{lstlisting}
				|\&是标准错误,即将标准错误加到他的标准输出中区,然后输出给下一个命令。time 关键字是用来输出时间的-p选项是用POSIX格式...
				\item 命令列表\\
				命令之间可以用|| 或\&\&或;或\&或者newline连接,这些连接符的优先即不同,; \&优先级最高、|| 和 \&\&优先级相等并低于前者。\&连接的命令将会匿名地异步的在后台执行。Shell并不会等待其完成。分号连接的命令按顺序执行。|| 连接的命令,命令2 仅当命令1执行且返回非0状态。\&\&连接的命令执行当且仅当命令1完成并返回0状态。
				\item 混合命令\\
				这是shell编程语言的构建,可以通过关键字来构建执行流。
				\begin{enumerate}
					\item 循环结构\\
					\begin{enumerate}
						\item uintil
\begin{lstlisting}[numbers = none]
	until test-commands; do consequent-commands; done
\end{lstlisting}
					当条件命令返回非0执行。
						\item while
\begin{lstlisting}[numbers = none]
	while test-commands; do consequent-commands; done
\end{lstlisting}	当条件命令返回0执行。					
						\item for
\begin{lstlisting}[numbers = none]
	for name [ [in [words ...] ] ; ] do commands; done 
\end{lstlisting}
				当name 值在括号中的列表中时,执行。
\begin{lstlisting}[numbers = none]
	for (( expr1 ; expr2 ; expr3 )) ; do commands ; done
\end{lstlisting}
				当expr2表达式不为0时执行(每执行一次检查)。
					\end{enumerate}
					\item 条件语句
					\begin{enumerate}
						\item if语句
\begin{lstlisting}[numbers = none]
	if test-commands; then
	consequent-commands;
	[elif more-test-commands; then
	more-consequents;]
	[else alternate-consequents;]
	fi
\end{lstlisting}
					条件命令返回0时执行。
					\item case 语句
\begin{lstlisting}[numbers = none]
	case word in
	[ [(] pattern [| pattern]...) command-list ;;]...
	esac
\end{lstlisting}
					匹配条件和执行命令通过)分割
					\item selects			
\begin{lstlisting}[numbers = none]
	select name [in words ...]; do commands; done
\end{lstlisting}		
					\item ((expression))
					\item (expression)
					\item ! expression
					\item expression1 \&\& expression2
					\item expression1 ||  expression2			
					\end{enumerate}
					\item 语句组
					\begin{enumerate}
						\item (list)
						\item (list;)
					\end{enumerate}
					\item 协处理
\begin{lstlisting}[numbers = none]
	coproc [NAME] command [redirections]
\end{lstlisting}	
				\end{enumerate}
			\end{enumerate}
			\item 函数
	\end{itemize}