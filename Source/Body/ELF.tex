\section{ELF文件的一些总结}
  \begin{quote}
    ELF主体结构为:文件头、程序头表、正文(包括重定位符号等)、区块表\cite{Yale}
  \end{quote}
  %\begin{figure}[ht]
  %  \centering
  %  \includegraphics[scale=0.6]{xiantu.pdf}
  %  \caption{ELF文件格式如图}
  %  \label{fig:xiantu}
  %\end{figure}
\begin{table}[h!]
  \centering
  \begin{tabular}{c|c|c}
    \hline
    Raw View & Linking View & Execution View \\
    \hline
    ELF header & ELF header & ELF header \\
    Program header table  &Program header table & Program header table \\
    - &optinal &  -\\
    data1  & Section 1 & \\
    ...  &...&Segment 1\\
     & Section n & \\
    data n  &...&Segment 2\\
    ...&...&...\\
    Section header table  &Section header table & Section header table \\
    -  &  - & optinal \\
    \hline
  \end{tabular}
\end{table}
\subsection{一些问题}
  \begin{itemize}
    \item 关于vmlinux.o符号表段中大段空白段数据,是预留给重定位地址的,
    因而可以在vmlinux文件中相同位置查看到有Value但是size为0的数据,即为重组vmlinx文件时用到的section地址 
  \end{itemize}
  
