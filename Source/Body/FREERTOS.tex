\section{移植FREERTOS的一些问题}
  \subsection{Segger}
    \begin{itemize}
      \item 打FreeRTOS补丁的格式注意2为忽略路径的层数、可以在git bash中运行
      \begin{lstlisting}
        Patch -p2 < *.patch
      \end{lstlisting}
      \item 使用Systemview时需要注意:SEGGER\_SYSVIEW\_Start()需要引用ul开头的一个变量来核对中断优先分组,
      但这个变量的赋值要在内核开始调度后重新计算赋值,因此要么把这个函数放在设置中
      断组之前要么在内核开始后执行
      \begin{enumerate}
        \item 在main.c 中添加头文件\#include "SEGGER\_SYSVIEW.h";
        \item 在main函数开始处添加:SEGGER\_SYSVIEW\_Conf(),初始化SystemView;
        \item 在FreeRTOS.h中添加头文件\#include "SEGGER\_SYSVIEW\_FreeRTOS.h";
        \item 在 FreeRTOSConfig.h  中添加两个宏定义:\\
        \#define INCLUDE\_xTaskGetIdleTaskHandle 1 \\
        \#define INCLUDE\_pxTaskGetStackStart    1 \\
      \end{enumerate}
      \item 剩下就是根据配置,增加一些相关的hook函数,根据编译报错
      \item 使用SEGGER\_SYSVIEW\_FreeRTOS.h时要有追踪宏定义的函数重定义问题,只需删去task文件中
     的宏定义参数或者修改SEGGER\_SYSVIEW\_FreeRTOS.h,在宏函数中添加一个参数即可
    \end{itemize}
  \subsection{关于FREERTOS中断等级的问题}
  \subsection{Keil的相关问题}
   主要是C库和printf重定向的问题,当文件使用库或者micro库时,
   再前添加SEGGER\_KEIL.C文件时会出现SYS\_IO.o 
   文件重定义的问题,这时候要么选择删去该文件(这样printf不能被重定向,只能使用RTT\_prrntf)
   要么取消micro并且重定向C库中的IO,重定向的方法是增加retarget\_io.c文件,可以在keil的包管理文件中,
    Compiler->I/O->STDOUT选择USR。这样工程文件中会自动出现retarget\_io.c文。