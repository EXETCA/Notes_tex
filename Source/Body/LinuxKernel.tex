\section{学习linux内核的一些笔记}
  \subsection{启动后的初始化}
    \begin{itemize}
      \item 金丝雀(Canary)\\
      这是一种防止栈溢出的手段。通过在栈尾增加一段特殊的溢出缓存空间(填充了特殊代码),通过每次执行函数后检查该空间,保证栈没有溢出。
      \item 锁核(信号量)然后在setup\_arch中实现具体的关于架构的设置\\
      	\begin{enumerate}
      		\item[] 从原来BIOS的信息中提取一些信息、包括根目录设备号,驱动信息,载入类型等,获取内核镜像的相关区块信息。
      		\item[] 设置内存域(物理内存)从BIOS中获取并设置全局变量,并映射代码段、数据段,也就是要读取镜像中的符号表。
      		\item[] 解析内核启动前的命令行输入,即启动内核配置。
      		\item[] 初始化多核结构体。
      		\item[]	 初始化虚拟地址映射,设置可分配的虚拟内存的初值。
      		
      	\end{enumerate}
    \end{itemize}