\section{MakeFile笔记}
  \subsection{语法与环境}
  \begin{itemize}
  	\item 变量\emph{VPATH}\
  		提供目标文件的搜素路径,但是脚本运行的路径不由该变量指定。\\
  		具体语法:
\begin{lstlisting}
VPATH = ...
\end{lstlisting}
	\item \emph{path}\\
	匹配模式下的目标搜索路径。\\
	具体语法:
\begin{lstlisting}
vpath pattern directory-list
\end{lstlisting}
	实例:
\begin{lstlisting}
vpath %.c src
vpath %.h include
\end{lstlisting}
	\item 通配符\\
		与Bourne Shell保持一致
		\begin{enumerate}
			\item \textasciitilde 表示当前用户的home路径
			\item * 代替0、单个、多个字符
			\item ? 代替一个字符
			\item \,\![ ... ]\:表示匹配方括号里的字符\\
				例如 [12345],表示只能匹配1、2、3、4、5中的任意一个字符
			\item {[\textasciicircum...]表示不匹配方括号里的字符}
%			\item \% 为Makefile规则通配符,等效于*,可以出现在规则匹配中的任何地方,但是只能出现一次
		\end{enumerate}
	\item 自动变量\\
		\begin{enumerate}
			\item \$@
			目标文件的文件名
			\item \$\%
			目标时数据库时,表示其中的成员的文件名
			\item \$<
			第一个依赖文件的文件名
			\item \$?
			表示比(所有)目标更新的依赖文件文件名
			\item \$\^
			表示所有依赖的文件名,这些文件名被空格隔开(不包括重复的)
			\item \$+
			表示所有依赖的文件名,这些文件名被空格隔开(包括重复的)
			\item \$*
			匹配模式中的,主文件名\%(不包括.c等后缀)
		\end{enumerate}
		\item 静态匹配\
		在特定的文件列表中匹配:
\begin{lstlisting}
$(OBJECTS):%.o:%.c
	$(CC) -C $(CFLAGS) $< -o $@
\end{lstlisting}
		\item 反直觉的后缀匹配模式\\
		由连续的两个后缀构成,其依赖在目标的前面,示例:
\begin{lstlisting}
.c.o:
	$(COMPILE.c) $(OUTPUT_OPTION) $< 
\end{lstlisting}
		\item 单后缀匹配\\
		一般用于生成可执行文件,unix系统中可执行文件一般不带后缀。
		\item 可用于匹配模式的常见后缀,由系统变量\$(.SUFFIXES)定义\\
		.SUFFIXES : .out .a .ln .o .c .cc .C .cpp .p .f .F .r .y .l \\
		通过重定义该变量可实现修改匹配后缀
		\item GNU Make 默认参数(潜规则)\\
		GNU Make 自带了很多匹配模式的规则,因此可以直接执行,或者仅仅对其中一些变量进行定义后就可执行,
		具体的规则可以使用下列命令查看
\begin{lstlisting}
make --print-data-base(-p)
\end{lstlisting}
		\item 环境变量作为参数传给make,注意环境变量等号之间不能由空格
\begin{lstlisting}
	make $(OBJECT) ENVERIONMENT=VALUE
\end{lstlisting}
		\item makefile的文件名必须时全小写或者Makefile才能被make软件默认识别。
		\item makefile里的目标文件必须加:才能被识别为目标,才能被默认执行。
		\item 特殊目标\\
			\begin{enumerate}
				\item .INTERMEDATE \\
				该目标文件的依赖会被当作中间文件,原来不存在,最后在make退出时删除,原来存在则不会删除。
				\item .SECONDARY \\
				依赖文件不会被删除。
				\item .PRECIOUS
				若执行被中断,则会删除其更新的目标文件。
				\item .DELETE\_ON\_ORROR\\
				若执行出错,只删除这个目标。
				\item .EXPORT\_ALL\_VARIABLES
			\end{enumerate}
		\item 自动依赖解析\\
		特别是针对标准库头文件的解析。利用GCC等编译器自带的头文件解析工具就可以解析除该头文件需求。\\
\cite[page.37]{MecklenburgNovember2004}
\begin{lstlisting}
$ echo "#include <stdio.h>" > stdio.c
$ gcc -M stdio.c
	stdio.o: stdio.c /usr/include/stdio.h /usr/include/_ansi.h \
	/usr/include/newlib.h /usr/include/sys/config.h \
	/usr/include/machine/ieeefp.h /usr/include/cygwin/config.h \
	/usr/lib/gcc-lib/i686-pc-cygwin/3.2/include/stddef.h \
	/usr/lib/gcc-lib/i686-pc-cygwin/3.2/include/stdarg.h \
	/usr/include/sys/reent.h /usr/include/sys/_types.h \
	/usr/include/sys/types.h /usr/include/machine/types.h \
	/usr/include/sys/features.h /usr/include/cygwin/types.h \
	/usr/include/sys/sysmacros.h /usr/include/stdint.h \
	/usr/include/sys/stdio.h
\end{lstlisting}
	如何将该解析结果加入Makefile?\\
\cite[page.39]{MecklenburgNovember2004} 
\begin{lstlisting}
depend: *.c *.c
	$(CC) -M $(CPPFLAGS) $^ > $@
include depend
\end{lstlisting}
	即将输出保存成文件、在包含这个文件的方式。
\begin{lstlisting}
sed 's,\($*\)\.o[ :]*,\1.o $@ : ,g' < $@.$$$$ > $@;
\end{lstlisting}
	其中s表示字符串替换模式\\
	\textbackslash (\$*\textbackslash)\textbackslash.o[ :]*首先make会替换\$,然后剩下的为正则表达式匹配内容,
	替换成\textbackslash 1.o \$@ : 这里\textbackslash1 会匹配前面的\textbackslash (\$*\textbackslash),make会将\$@换成目标文件文件名,最后一个参数g表示整行匹配替换,若没有这个g,则一行中出现多次匹配内容时,他仅仅匹配一次
	\item 静态库操作\\
		\begin{enumerate}
			\item ar 指令\\
			ar rv lib.a object.o 将object.o 打包到lib,a中
			\item 链接静态库\\
			cc object.o lib1.a lib2.a -o object 即链接objet\\
			cc object.o -l1 -l2 -o object 简写形式并且注意链接库的lib前缀去掉了并增加了-l\\
			cc object.o -L.  -l1 -l2 -o object 使用-L修改搜素路径、
			\item 利用make简化工作\\
			\cite[page.43]{MecklenburgNovember2004} 
\begin{lstlisting}
libgraphics.a(bitblt.o): bitblt.o
	$(AR) $(ARFLAGS) $@ $<
\end{lstlisting}			
		\end{enumerate}
	\item 双冒号规则\\
	即同一个目标文件更具不同依赖对象更新执行不同的指令
\begin{lstlisting}
file-list:: generate-list-script
	chmod +x $<
	generate-list-script $(files) > file-list
file-list:: $(files)
	generate-list-script $(files) > file-list
\end{lstlisting}
 \end{itemize}
  \subsection{变量与宏定义} 、
	\begin{itemize}
		\item 变量的命名规则\\
		绝大部分字符都支持,只有:,\#,=这三个符号不支持,大小写敏感,要获取变量的值必须用\$()包围他,如果时单字符可以省略括号,这就是为什么自动变量不用带括号的原因。
		\item 变量的定义\\
		注意变量末尾的空格不会自动忽略,因此要注意,当变量未定定义被使用,则变量会拓展为空格而不是报错\\
		类型: 简单拓展类型用:=定义;递归类型用=定义,即变量可以不安顺序定义,并且每次使用递归变量时,变量会实时变动;条件变量,用?=定义,当变量未被定义,再定义;递增定义+=,增加变量的字符,=定义的变量无法自增不然会陷入循环,因此增加了该符号。
		\item 宏定义\\
		可以说宏定义就是多行的变量,因为宏定义和变量展开方式都是字符串展开。宏定义以define开头以endef结尾,注意必须换行。
\begin{lstlisting}
define MACROS_NAME
	BODY...
	...
	...
endef
\end{lstlisting}
		使用宏定义时在前面增加@,可以使得每一行都固定增加@,用来显示执行命令。
	 \item 变量的扩展规则\\
	 	make工作时会执行两轮,第一轮扫描makefile并执行include、保存变量、依赖关系到数据库中,第二轮make将会分析依赖并决定更新。当宏定义扩展时,被扩展的文字也会被立刻被递归扩展或者引用变量,如果宏定义在命令中扩展则会自动增加制表符。
	 	\begin{enumerate}
	 		\item 对于变量赋值语句的左边变量,第一遍执行make时,make将总是立刻赋值。(因此左边如果存在表达式,必须由前置的赋值语句,否则使用上述的空规则,即给空赋值)
	 		\item =或?=的右边,将在第二遍执行时赋值。因此右侧的表达式如果含有变量,则以最终的赋值语句为准。
	 		\item :=的右边将会立刻扩展。因此右边的表达式上面一定要定义一次。
	 		\item += 如果左时简单的立即变量(由:=定义)将会立刻扩展,否则将会被延迟。
	 		\item 宏定义名字将会被立刻展开,但是定义内容将会延迟。
	 		\item 目标和依赖将会立刻展开,但是命令会被延迟。
	 	\end{enumerate}
	 	\item 仅对某目标文件有效的变量定义\\
\begin{lstlisting}[numbers = none]
target...: variable = value
target...: variable := value
target...: variable += value
target...: variable ?= value
\end{lstlisting}	 	
		\item 变量传递\\
		环境变量传递到make,一般而言makefile里的变量会将环境变量覆盖、但是通过--enviroment-overrides(or -e)选项可以使得环境变量覆盖makefile的变量。\\
		当make递归调用时,变量可以通过环境变量传递给子make。通过export 标记可以将makefile中的变量传递到环境变量。通过unexport 标记阻止这种传递。
	\end{itemize}
\subsection{条件语句}
\begin{itemize}
	\item 语法\\
\begin{lstlisting}
if-condition
text if the condition is true
else
text if the condition is false
endif
\end{lstlisting}	
	\item  	if-condition\\
	\begin{enumerate}
		\item ifdef variable-name
		\item ifndef variable-name
		\item ifeq test
		\item ifneq test
	\end{enumerate}
	variable-name不需要用\$(),test 可以是"a" "b" 或者 (a,b)
\end{itemize}
\subsection{包含include}
\begin{itemize}
	\item include filename1 ... (or wildcards)
	注意make,第一遍找不到include时,make会跳过执行,当全部依赖建立完后,make会尝试匹配include文件和依赖,若include文件更新,make将会重新启动make再次执行。再次执行还是没有inlcude,那么make就会出错。
\end{itemize}
\subsection{标准变量}
\begin{enumerate}
	\item MAKE\_VERSION\\
	显然是make的版本
	\item CURDIR\\ 
	当前工作目录。但是当make启用-directory=directory-nameor-Cdirectory-name选项
	\item MAKEFILE\_LIST\\
	加载的makefile的文件名
	\item MAKECMDGOALS\\
	make的目标,特别是从命令行加载的目标
	\item .VARIBLES\\
	包含了到引用变量之前的在makefile中定义的变量
\end{enumerate}
\subsection{函数}
	\begin{itemize}
		\item 用户定义函数\\
		带参数的宏定义、宏名可以使参数也可以是变量(宏就是一个多行的变量)
\begin{lstlisting}[numbers = none]
$(callmacro-name[,param1...])
\end{lstlisting}	 	
		参数1,不能是带\$符号的变量,因为make将会将其值传递。如果宏没有引用参数,而你使用call传递了参数,这并不会发生什么。如果宏定义中嵌套调用了宏,那么子宏的引用的参数会被显示到父中
		\item make 内置的函数\\
		\begin{enumerate}
			\item filter
\begin{lstlisting}[numbers = none]
	$(filterpattern...,text)
\end{lstlisting}				
			从TEXT中过滤除符合pattern的字符,注意pattern中只能有一个\%,若2则第一个会被视为字符\%。\\
			衍生:filter-out 取不符合的字符 
			\item findstring
\begin{lstlisting}[numbers = none]
	$(findstring string,text)
\end{lstlisting}
			找到则返回,找不到则返回空。但是string中不能包含通配符
			\item subst 
\begin{lstlisting}[numbers = none]
	$(substsearch-string,replace-string,text)
\end{lstlisting}			
			简单的非通配搜素、替换。
			\item patsubst 
\begin{lstlisting}[numbers = none]
	$(patsubst search-pattern,replace-pattern,text)
\end{lstlisting}
			单\%匹配。
			简写形式
\begin{lstlisting}[numbers = none]
	$(variable:search=replace)
\end{lstlisting}
			\item words 
\begin{lstlisting}[numbers = none]
	$(words text)
\end{lstlisting}
			返回text中的词的个数。
\begin{lstlisting}[numbers = none]
	$(word n,text)
\end{lstlisting}
			返回text中第N个词
\begin{lstlisting}[numbers = none]
	$(firstword text)
\end{lstlisting}
			返回第一个词
\begin{lstlisting}[numbers = none]
	$(wordlist start,end,text)
\end{lstlisting}			
			返回第n-m个词
			\item sort
\begin{lstlisting}[numbers = none]
	$(sort list)
\end{lstlisting}	
			将字符按顺序排列
			\item shell
\begin{lstlisting}[numbers = none]
	$(shell command)
\end{lstlisting}	
			将命令放到子shell中运行,并返回结果
			\item wildcard 文件通配指令
\begin{lstlisting}[numbers = none]
	$(wildcard pattern...)
\end{lstlisting}			
			注意该函数用于搜素文件,与\$(if)联动可以判断文件是否存在
			\item dir 
\begin{lstlisting}[numbers = none]
	$(dir list...)
\end{lstlisting}		
			返回列表中的所有目录	
			\item notdir
\begin{lstlisting}[numbers = none]
	$(nodir name...)
\end{lstlisting}			
			返回文件路径中的文件名部分
			\item suffix
\begin{lstlisting}[numbers = none]
	$(suffix name...)
\end{lstlisting}			
			返回各个词的前缀	
			\item strip
\begin{lstlisting}[numbers = none]
	$(strip text)
\end{lstlisting}
			用于删除text中的开头和尾巴处的空格字符,并将内部空白(多个空格)合并为一个空格。
		\end{enumerate}
		\item 流控制函数
			\begin{enumerate}
				\item if 
\begin{lstlisting}[numbers = none]
	$(if condition,then-part,else-part)
\end{lstlisting}
				\item error 
\begin{lstlisting}[numbers = none]
	$(error text)
\end{lstlisting}
				\item foreach
\begin{lstlisting}[numbers = none]
	$(foreach variable,list,body)
\end{lstlisting}
			执行用list的元素替换varibale 然后展开成body,直到list中的元素被替换完位置。
				
			\end{enumerate}
	\end{itemize}